\documentclass[letterpaper, twocolumn]{article}
% \usepackage{usenix2019_v3}

% to be able to draw some self-contained figs
\usepackage{tikz}
\usetikzlibrary{shapes.geometric}
\usepackage{amsmath}
\usepackage{url}
\usepackage{algpseudocode}
\usepackage{algorithm}

% inlined bib file
% \usepackage{filecontents}

\AtBeginDocument{%
  \providecommand\BibTeX{{%
    Bib\TeX}}}

%-------------------------------------------------------------------------------
\begin{document}
%-------------------------------------------------------------------------------

%don't want date printed
\date{}

\subsection{Eviction Sets}
Eviction sets are an attack primitive used in many cache-based hardware attacks.
They are used to prime a cache by ensuring that a target object is removed from its cache set.
An eviction set is a group of virtual addresses that are all congruent: they all map to the same cache set.
For a cache set with associativity $a$, an eviction set $S$ must contain $|S| \geq a$ addresses.
In a system without pseudo-random mapping of addresses to cache sets, congruent addresses share index bits in their
corresponding physical address.

% An attacker may use an eviction set to fill a target cache set with attacker-provided objects; we refer to this process as \textit{occupation} of an eviction set.

Eviction sets are used in the following steps:

\begin{enumerate}
    \item \textit{Creation:} An attacker finds a set of congruent virtual addresses $S$ for $a$-associativity cache with $|S| \geq a$. Optionally, the attacker may choose a minimal eviction set with $|S|=a$.
    Although minimizing the eviction set can simplify the accurate detection of information leakage,
    many attacks are successful with eviction sets larger than their target caches.

    \item \textit{Occupation:} An attacker touches the memory addresses in the eviction set until the target cache is entirely filled with memory addresses from the eviction set.
    Anything previously in the cache set should be evicted through this process.
    This step is often called \textit{priming the cache} – we refer to this process as cache set \textit{occupation}.
\end{enumerate}

In rowhammer attacks, eviction sets are used to ensure that an attacker's memory request reaches DRAM.
The attacker repeatedly touches a target address, evicts the target address so subsequent accesses will not be caught by the cache,
and re-references the target address.

In transient attacks, a target address is usually evicted as part of a timing attack, as a memory request to an evicted object
must take time to travel to and from DRAM, and this timing difference is significant enough to be measured
by high-performance counters or simply by a continuously incrementing variable.
In Prime+Probe, the attacker occupies the cache set of a target address, executes the victim program, and re-references
all the addresses in the eviction set; an increase in the time taken implies that the target address was referenced and evicted
one of the objects of the eviction set.
In Evict+Reload and Evict+Time, the attacker measures the victim's runtime before and after occupying a cache set,
with a longer re-run time implying a reference of an evicted cache object.

We primarily aim to slow cache set \textit{occupation}.
Most eviction set attacks create eviction sets once but use them for occupation many times;
by slowing occupation, we reduce the information leakage throughput of these attacks
and possibly reduce the likelihood of rowhammer bitflips.

\section{Design}

We seek to make rowhammer-style and transient execution attacks more difficult by preventing or slowing the use of eviction sets
using hardware-level cache eviction algorithms.
Additionally, we aim to accomplish this security without inducing a significant performance hit in the cache of either miss rate or latency.

\subsection{Expected Difficulty of an Eviction Algorithm}
We define the difficulty $D$ of occupying a cache set with a certain touching pattern
as the total number of memory touches required to overwrite all lines in the cache
with addresses from an eviction set.

For the minimum difficulty $D_m$, we assume an optimal attacker;
for each eviction algorithm, there exists a memory touching pattern that
most effectively primes a cache set.
In this case, an optimal attacker does not know the state of the cache,
but may optionally use a timing oracle to determine if a previously touched address
is still in the cache.
This timing oracle incurs an extra counted touch, but we ignore the cost of
the added conditional.

Additionally, randomness is a factor in some of these eviction algorithms.
We assume a uniform random distribution of initial cache states;
We define the expected minimum difficulty $D_{Em}$ of an eviction algorithm
as the mean number of optimally-ordered touches required to fill a cache set.

Finally, SLRU and our implementation of 3Tree splits the cache into segments,
which may mean that the attacker can guarantee that a victim overwrites
an attacker-primed address without the attacker filling the entire cache set,
and thus that the minimum expected difficulty may be higher than the true value
when used in an attack.
We define the guarantee difficulty $D_g$ as the minimum expected number of touches
required to guarantee an attacker access to an accurate timing oracle.
For replacement algorithms where new objects may be placed in any index in the cache,
$D_g = D_{Em}$, as an attacker must fill every cache line to guarantee
that a victim touch will overwrite an eviction set address.

\subsubsection{LRU, TreePLRU, and FIFO}
% Derive expected difficulty, no choice or permutation randomness
LRU, TreePLRU, and FIFO pose little defense against cache priming.
In both FIFO and LRU caches, new cache objects are instantly promoted to the head of the queue.
In LRU, new objects are marked as the most recently updated and thus the last to be evicted,
and the FIFO queue means new objects are evicted after objects added earlier.
Although TreePLRU does not follow the same linear structure,
it is similarly trivial to occupy a cache set.
Thus, for an $a$-way associativity LRU, FIFO, or TreePLRU cache, an attacker only needs
$a$ sequential touches to fill a cache set.
As these algorithms are deterministic, the expected difficulty is the same value.

\begin{equation}\label{LRUExpectedD}
  D_g(\text{LRU}) = D_m(\text{LRU}) = D_{Em}(\text{LRU}) = a
\end{equation}

\subsubsection{Random Replacement}
Given a cache set with associativity $a$ in which an attacker has occupied $n$ addresses
using some subset $S \subset E, |S| = n$ of eviction set $E$,
a following touch on address $\alpha$ may be in $S$ or not.

If $\alpha \in S$, there is no cache or per-line metadata to update, and the cache state does not change.

If $\alpha \notin S$, the random replacement algorithm will evict an address $\beta$.
This evicted address will occupy an unoccupied line with probability
$P(\beta \notin S) = \frac{a-n}{a}$.

Then, the probability occupying $n+1$ addresses in a cache set with $n$ addresses occupied is $P(n+1) = \frac{a-n}{a}$.
The corresponding expected value is $EV(n+1) = \frac{1}{P(n+1)} = \frac{a}{a-n}$
Then, the expected value of the total number of unique touches to occupy an eviction set is

% If they have an issue with this not being fully simplified they can take that up in review
\begin{equation}\label{RandomExpected}
    D_g(\text{Random}) = D_{Em}(\text{Random}) = \sum_{i=0}^{a-1}{\frac{a}{a-i}}
\end{equation}

\subsection{Preventing Eviction Set Occupation}
Attackers use eviction sets as attack primitives to guarantee that a target address is not present in the cache
so a victim's memory request of the target address must travel to DRAM and back.
Leaving at least one of the lines in a cache set open and unoccupied means that attackers have no guarantee that the target address is evicted.
Thus, to slow the use of an eviction set in occupying a cache set, we must only increase the difficulty of the eviction algorithm used.

\subsection{Latency and Energy Costs}

As the CPU implements eviction algorithms in hardware, the function of some software-based algorithms
may not be exactly reflected without inducing significant performance or complexity overhead.
For example, LRU can be implemented in a software cache to add, update, and remove objects in $O(1)$ time.
Although no such time complexity exists for algorithm circuitry that runs in parallel,
exact LRU is complex in CPU caches and involves lots of metadata per cache line and lots of circuitry to
determine an eviction candidate.
Excessive circuitry introduces extra latency by increasing the critical path of the circuit
and extra energy usage by requiring additional transistors.
The LRU algorithm is thus commonly simplified to TreePLRU in the CPU, which requires much less hardware complexity
but achieves a similar miss ratio.
Although the energy usage can be significant, modern CPU features like prefetching largely

%%%%%%%%%%%%%%%%%%%%%%%%%%%%%%%%%%%%%%%%%%%%%%%%%%%%%%%%%%%%%%%%%%%%%%%%%%%%%%%%
\end{document}
%%%%%%%%%%%%%%%%%%%%%%%%%%%%%%%%%%%%%%%%%%%%%%%%%%%%%%%%%%%%%%%%%%%%%%%%%%%%%%%%

%%  LocalWords:  endnotes includegraphics fread ptr nobj noindent
%%  LocalWords:  pdflatex acks
