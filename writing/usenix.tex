%%%%%%%%%%%%%%%%%%%%%%%%%%%%%%%%%%%%%%%%%%%%%%%%%%%%%%%%%%%%%%%%%%%%%%%%%%%%%%%%
% Template for USENIX papers.
%
% History:
%
% - TEMPLATE for Usenix papers, specifically to meet requirements of
%   USENIX '05. originally a template for producing IEEE-format
%   articles using LaTeX. written by Matthew Ward, CS Department,
%   Worcester Polytechnic Institute. adapted by David Beazley for his
%   excellent SWIG paper in Proceedings, Tcl 96. turned into a
%   smartass generic template by De Clarke, with thanks to both the
%   above pioneers. Use at your own risk. Complaints to /dev/null.
%   Make it two column with no page numbering, default is 10 point.
%
% - Munged by Fred Douglis <douglis@research.att.com> 10/97 to
%   separate the .sty file from the LaTeX source template, so that
%   people can more easily include the .sty file into an existing
%   document. Also changed to more closely follow the style guidelines
%   as represented by the Word sample file.
%
% - Note that since 2010, USENIX does not require endnotes. If you
%   want foot of page notes, don't include the endnotes package in the
%   usepackage command, below.
% - This version uses the latex2e styles, not the very ancient 2.09
%   stuff.
%
% - Updated July 2018: Text block size changed from 6.5" to 7"
%
% - Updated Dec 2018 for ATC'19:
%
%   * Revised text to pass HotCRP's auto-formatting check, with
%     hotcrp.settings.submission_form.body_font_size=10pt, and
%     hotcrp.settings.submission_form.line_height=12pt
%
%   * Switched from \endnote-s to \footnote-s to match Usenix's policy.
%
%   * \section* => \begin{abstract} ... \end{abstract}
%
%   * Make template self-contained in terms of bibtex entires, to allow
%     this file to be compiled. (And changing refs style to 'plain'.)
%
%   * Make template self-contained in terms of figures, to
%     allow this file to be compiled. 
%
%   * Added packages for hyperref, embedding fonts, and improving
%     appearance.
%   
%   * Removed outdated text.
%
%%%%%%%%%%%%%%%%%%%%%%%%%%%%%%%%%%%%%%%%%%%%%%%%%%%%%%%%%%%%%%%%%%%%%%%%%%%%%%%%

\documentclass[letterpaper,twocolumn,10pt]{article}
\usepackage{usenix2019_v3}

% to be able to draw some self-contained figs
\usepackage{tikz}
\usepackage{amsmath}

% inlined bib file
\usepackage{filecontents}

%-------------------------------------------------------------------------------
\begin{filecontents}{\jobname.bib}
%-------------------------------------------------------------------------------
@Book{arpachiDusseau18:osbook,
  author =       {Arpaci-Dusseau, Remzi H. and Arpaci-Dusseau Andrea C.},
  title =        {Operating Systems: Three Easy Pieces},
  publisher =    {Arpaci-Dusseau Books, LLC},
  year =         2015,
  edition =      {1.00},
  note =         {\url{http://pages.cs.wisc.edu/~remzi/OSTEP/}}
}
@InProceedings{waldspurger02,
  author =       {Waldspurger, Carl A.},
  title =        {Memory resource management in {VMware ESX} server},
  booktitle =    {USENIX Symposium on Operating System Design and
                  Implementation (OSDI)},
  year =         2002,
  pages =        {181--194},
  note =         {\url{https://www.usenix.org/legacy/event/osdi02/tech/waldspurger/waldspurger.pdf}}}
\end{filecontents}

%-------------------------------------------------------------------------------
\begin{document}
%-------------------------------------------------------------------------------

%don't want date printed
\date{}

% make title bold and 14 pt font (Latex default is non-bold, 16 pt)
\title{\Large \bf SIEVE in the CPU:\\
  Securing the Cache against Primed Side Channel Attacks}

%for single author (just remove % characters)
\author{
{\rm Rishav Chakravarty}\\
Dartmouth College
\and
{\rm Second Name}\\
Second Institution
% copy the following lines to add more authors
% \and
% {\rm Name}\\
%Name Institution
} % end author

\maketitle

%-------------------------------------------------------------------------------
\begin{abstract}
%-------------------------------------------------------------------------------
Your abstract text goes here. Just a few facts. Whet our appetites.
Not more than 200 words, if possible, and preferably closer to 150.

I'll write this when I'm done with the rest of the paper as a summary of everything
\end{abstract}


%-------------------------------------------------------------------------------
\section{Introduction}
%-------------------------------------------------------------------------------

%-------------------------------------------------------------------------------
\section{Background}
%-------------------------------------------------------------------------------

\subsection{Caches and Eviction Algorithms}

Caches are memory modules between processors and larger, slower memory modules.
They trade size for speed; they usually contain a subset of the data of main memory that is most referenced but,
due to their size, may be physically closer to a processor and greatly reduce latency.
Caches are used in many computing environments where the latency of memory is a significant factor in performance,
such as small databases between clients and CDNs, and in CPUs between the ALU and DRAM.

(This isn't complete - something about higher level groups than sets, too)
Modern CPUs use multiple levels of caches. In order, the L1, L2 and optional L3 cache are increasingly large and far from the ALU.
The Last Level Cache (LLC) is often shared among all CPU cores, and each CPU core contains one of each of the smaller cache levels.
Each cache is divided into sets, which each contain cache lines.
The number of cache lines in each set is called the associativity of the cache;
cache sets range from one line each (Direct Mapped) to the size of the entire cache (Fully Associative),
although 2, 4, 8, and 16-way associativity caches are most common in CPUs.
Caches close to the ALU tend to have lower associativity values of 2 or 4, and caches farther away
tend to have higher values between 4 and 16.
Each cache line is typically 64 bytes long.

As caches have a limited size, replacement policies, also known as eviction algorithms,
decide based on cache and per-object metadata which eviction candidates to remove to make space for new objects.
The ideal eviction algorithm minimizes the number of cache misses, where a memory requests data not in the cache
and must wait much longer for a response from main memory.
An efficient eviction algorithm will retain objects that are commonly used or predicted to be used again in the near future,
and will evict objects that are stale or not predicted to be used again.
The miss ratio describes the efficiency of a cache:
\begin{math}
  miss\ rate = \frac{misses}{hits + misses}
\end{math}

A cache may also be measured by its latency, which describes the amount of time taken to fulfill a memory request.
**Equation: cache request time**
In a CPU cache, this is often measured as the time of the critical path of the circuit used to compute an eviction candidate.

\subsubsection{TreePLRU}

The Least Recently Used (LRU) algorithm is a common algorithm for caching eviction.
Upon cache access, it promotes the requested memory to the Most Recently Used (MRU) in a queue;
upon cache miss, it evicts the least recently used cache line from the end of this queue.
However, in the hardware of a CPU, this algorithm introduces considerable complexity;
the optimal implementation involves checking the cache state against each permutation of recency values to determine an eviction candidate.

TreePLRU is a type of Pseudo-Least Recently Used (PLRU) algorithm that is very commonly used in CPU caches
because it closely replicates the functionality of LRU with much less hardware complexity.
It uses a tree structure to recursively separate hot object groups from cold groups.
Each node in the tree uses one bit of metadata to point to either the left or right subtree; the whole tree for an a-way associativity cache
only uses a-1 bits of metadata.
Upon a cache hit, an object is promoted by tracing up the tree and setting the bit to point away from the hit object;
upon a cache miss, the tree is recursively traced from the root following the direction of the bits, and the
leaf node reached at the end is the cache line chosen as the eviction candidate.

**Diagram**

TreePLRU is commonly used in CPUs in L2 and (if the processor has it) larger L3 caches, as it is relatively simple to implement 
and shows very good miss rates for common workloads.

\subsubsection{Random Replacement}
Random Replacement is an eviction algorithm sometimes used in CPU caches.
Upon a cache miss, this algorithm simply chooses a cache line to evict at random.
As this algorithm does not attempt to evaluate cache lines for recency of use or predict their next use,
it suffers from a higher miss ratio than many other algorithms.
However, since it requires no cache metadata and no algorithm logic to decide on an eviction candidate,
it can have incredibly low latency.
For this reason, Random Replacement is sometimes used in L1 caches, which have lower associativity
and are thus less prone to (**what is it called when multiple addresses map to the same set?**).
The low latency of these caches - often around **X** clock cycles to fulfill a memory request -
minimizes energy usage and the expected response time (*reference figure*).

\subsubsection{TwoQ}
TwoQ is a cache eviction algorithm developed for large software caches like web and database caches.
It improves upon LRU-k, which is a variant of LRU that evicts cache objects based on their kth previous access.

(figure)

TwoQ uses two queues, $Am$ and $A1$, to separate hot objects from cold ones.
Upon a cache miss, new objects are added to the $A1$ FIFO queue.
When rereferenced, they are moved to the head of $Am$, which is an LRU queue.
Cache objects move from $A1$ to $Am$ to signify that they are hot objects, and are never moved back to $A1$ –
they are removed when they reach the end of the LRU queue.

A single buffer is subdivided to make $Am$ and $A1$.
The algorithm requires tuning of the sizes of these subdivisions to make the cache as efficient as possible;
if $Am$ is too small, the cache will not be able to hold hot objects for long enough before they are evicted,
and if $A1$ is too small, certain access patterns will evict cold objects before they have a chance to be marked as hot.
The original paper finds that an $Am$ three times as large as $A1$ (75\%-25\% split) is optimal for certain workloads.

This version of the algorithm performs poorly in real traces of large software caches because
it struggles with traces in which the locality changes quickly.
A 'full' version of the algorithm with a third queue, $A1out$, is proposed to solve this problem.
However, changing locality is not as much of an issue in CPU caches as in large software caches.
If a popular set of addresses suddenly becomes unpopular compared to another set of addresses,
the new set of addresses will most likely map to a different cache set and not interfere with the previous addresses.
For the sake of simplicity, and given that small CPU caches would suffer from a small $Am$ if split among 3 queues,
we ignore the TwoQ full version and focus on the simple version.

\subsubsection{SIEVE}

SIEVE is a simple algorithm used for large software caches as both an eviction algorithm
and as a cache primitive used as a foundation for other, more complicated algorithms.
It improves upon LRU and FIFO Reinsertion,
in which elements are placed in a FIFO queue with a \textit{safe} boolean:
upon a touch, the element is marked as safe, and upon eviction,
the first non-safe object is evicted.

SIEVE uses only one booleans of metadata per cache line, \textit{visited}
as well as a \textit{hand} to trace through the cache.
When touched, cache objects are marked as visited.
Upon eviction, the hand marches through the cache array, returning to the head if at the tail;
it evicts the first element that has its \textit{visited} bit unset,
and unsets the \textit{visited} bits of any cache objects it traces through along the way.
Instead of placing the new object at the eviction position,
the algorithm always places new objects at the head of the queue,
which prevents the new and the retained objects from being mixed together.
The hand persists its position between cache operations,
effectively resulting in a sliding window of objects at risk of eviction.

This algorithm shows significant average improvements in CDN workflow miss ratio
over many eviction algorithms,
including LRU, FIFO, ARC, and TwoQ, while being simpler than these alternatives.
It is based on principles of \textit{lazy promotion},
in which objects are only promoted at eviction time,
and \textit{quick demotion}, in which most new objects are removed quickly after their insertion.
Additionally, it significantly improves over the data usage of other
software caching algorithms, as the single boolean of metadata per line and the hand
are small and constant in memory size regardless of the capacity of the cache,
whereas timestamps for algorithms like LRU must each contain much more data,
and other index-based ordering schemes must increase their memory footprint for increasing cache size.
The algorithm also has an advantage in multithreaded software caches;
since only the visited bit of an object is set upon a cache object touch,
only cache evictions require a lock on the whole cache array.
For multithreaded applications, this vastly improves throughput,
as most other contemporary cache eviction algorithms require a whole-cache lock
both evictions and touches.

\subsection{Cache-enabled Attacks}
Caches are mostly well-protected against direct attacks;
users are prevented from accessing memory placed in the cache by another process.
However, the cache can serve as a tool to enable other attacks.

\subsubsection{Transient Attacks and Forms}
Transient attacks, also called speculative execution attacks, take advantage of modern CPUs' speculative execution pipeline.
Speculative execution increases performance by guessing uncertain values that take time to evaluate, such as branch destinations
and conditions, and continuing execution under the assumption of the guess;
if the guess is later found to be incorrect, the CPU flushes the pipeline and rolls back to the last checkpoint known to be correct.

The speculative window - the period between a condition starting speculative execution and its value confirmation -
introduces a vulnerability;
If the guessed condition was incorrect, malicious code may be run in the speculative window.
Many modern CPUs, prioritizing performance, roll back only as much data as is required to continue execution
with correct functionality; code in the speculative window can leave traces in a covert channel in hardware
components where speculative misses are not rolled back.

As outlined in [source], transient execution attacks take advantage of this speculative execution window in three steps:
\begin{enumerate}
    \item \textit{Setup Phase}:
An attacker primes the microarchitectural state such that victim actions can later be decoded and intrepreted
    \item \textit{Transient Execution Phase}:
Speculative execution takes place, usually induced in a victim process by the attacker using a disclosure gadget.
The malicious code runs in the speculative window, quietly leaving a trace in the primed covert channel.
This attacking code is then squashed by the speculative execution pipeline, but the trace remains.
    \item \textit{Decoding Phase}:
An attacker decodes the information leaked into the covert channel during the Transient Execution Phase.
\end{enumerate}

As the CPU cache is too large to fully roll back upon a speculative miss, this is a common side of transient information leakage.
Previous hardware solutions to transient attacks include MuonTrap,
which introduces an additional L0-level cache that catches all speculative memory accesses
and can easily roll back upon a speculative miss.

Spectre and Meltdown, with victim-executed speculative execution and exception-induced speculative execution respectively, 
are early examples of transient attacks.
New transient attacks like PACMAN, TikTag, GoFetch, and more, use other components and features of the CPU.
These attacks follow many different forms depending on the capabilities of the hardware they run on and the
nature of the process they seek to attack.
We focus on defending against the following forms:
\begin{enumerate}
    \item \textit{Prime+Probe}:
An attacker primes a known location in cache, executes a victim program, and determines 
if the victim program accessed the primed location. This is a very common and relatively simple
type of transient execution attack.
    \item \textit{Evict+Reload}:
An attacker loads a shared binary object into its address space and primes the cache for an address
in the shared binary.
After executing the victim program, the attacker determines if the victim program accessed the primed location.
    \item \textit{Evict+Time}:
An attacker first measures the execution time of a victim program.
After priming the cache, the victim process is run and timed again.
A timing difference implies that the victim process accessed the memory affected by the cache priming.
\end{enumerate}

These attack forms are applicable to transient attacks and many others to establish a covert channel in the cache for information leakage,
and they all use eviction sets to prime the cache.
We aim to prevent or slow the use of eviction sets in the \textit{setup phase} of the transient attack, which will decrease the throughput of information leakage.

\subsubsection{Rowhammer}

\subsection{Eviction Sets}
Eviction sets are an attack primitive used in many cache-based hardware attacks.
They are used to prime a cache by ensuring that a target object is removed from its cache set.
An eviction set is a group of virtual addresses that are all congruent: they all map to the same cache set.
For a cache set with associativity $a$, an eviction set $S$ must contain $|S| \geq a$ addresses.
In a system without cache scrambling (**check on the name of this + source**), congruent addresses share index bits in their
corresponding physical address. (**check on virtual vs. physical**)

% An attacker may use an eviction set to fill a target cache set with attacker-provided objects; we refer to this process as \textit{occupation} of an eviction set.

Eviction sets are used in the following steps:

\begin{enumerate}
    \item \textit{Creation:} An attacker finds a set of congruent virtual addresses $S$ for $a$-associativity cache with $|S| \geq a$. Optionally, the attacker may choose a minimal eviction set with $|S|=a$.
    Although minimizing the eviction set can simplify the accurate detection of information leakage,
    many attacks are successful with eviction sets larger than their target caches.

    \item \textit{Occupation:} An attacker touches the memory addresses in the eviction set until the target cache is entirely filled with memory addresses from the eviction set.
    Anything previously in the cache set should be evicted through this process.
    This step is often called \textit{priming the cache} – we refer to this process as cache set \textit{occupation}.
\end{enumerate}

In rowhammer attacks, eviction sets are used to ensure that an attacker's memory request reaches DRAM.
The attacker repeatedly touches a target address, evicts the target address so subsequent accesses will not be caught by the cache,
and re-references the target address.

In transient attacks, a target address is usually evicted as part of a timing attack, as a memory request to an evicted object
must take time to travel to and from DRAM, and this timing difference is significant enough to be measured
by high-performance counters or simply by a continuously incrementing variable.
In Prime+Probe, the attacker occupies the cache set of a target address, executes the victim program, and re-references
all the addresses in the eviction set; an increase in the time taken implies that the target address was referenced and evicted
one of the objects of the eviction set.
In Evict+Reload and Evict+Time, the attacker measures the victim's runtime before and after occupying a cache set,
with a longer re-run time implying a reference of an evicted cache object.

We primarily aim to slow cache set \textit{occupation}.
Most eviction set attacks create eviction sets once but use them for occupation many times;
by slowing occupation, we reduce the information leakage throughput of these attacks
and possibly reduce the likelihood of rowhammer bitflips.

\section{Design}

We seek to make rowhammer-style and transient execution attacks more difficult by preventing or slowing the use of eviction sets
using hardware-level cache eviction algorithms.
Additionally, we aim to accomplish this security without inducing a significant performance hit in the cache of either miss rate or latency.

\subsection{Expected Difficulty of an Eviction Algorithm}
We define the difficulty of occupying a cache set with a certain touching pattern as the total number of memory touches
required to overwrite all lines in the cache with addresses from an eviction set.
With this measurement, we evaluate the LRU (and its variants) and Random Replacement eviction algorithms for their security against cache set occupation.
We use these measurements as baselines of comparisons for our own cache eviction algorithm.

For the minimum difficulty, we assume an optimal attacker;
for each eviction algorithm, there exists a memory touching pattern that most effectively fills a cache set.
We define the minimum difficulty of an eviction algorithm as the number of optimally-ordered touches required to fill a cache set.

Additionally, randomness is a factor in some of these eviction algorithms.
We assume a uniform random distribution of initial cache states;
for some algorithms, the initial state into which an attacker would start placing eviction set objects affects the order of evictions.
Additionally, some algorithms use randomness in selecting an eviction candidate.
We define the expected minimum difficulty of an eviction algorithm as the mean number of optimally-ordered touches required to fill a cache set.

\subsubsection{LRU, TreePLRU, and FIFO}
These eviction algorithms pose little defense against cache occupation.
In both FIFO and LRU caches, new cache objects are instantly promoted to the head of the queue;
in LRU, new objects are marked as the most recently updated and thus the last to be evicted,
and the FIFO queue means new objects are evicted after objects added earlier.
[Diagram]
Although TreePLRU does not follow the same linear structure, it is similarly trivial to occupy a cache set.
[Diagram]
Thus, for an a-way associativity LRU, FIFO, or TreePLRU cache, an attacker only needs $a$ sequential touches to fill a cache set.

\begin{equation}
    \text{D}_{LRU}(a)=a
\end{equation}

As these algorithms are deterministic, the expected difficulty is the same value.
[Equation]

\subsubsection{Random Replacement}
Given a cache set with associativity $a$ in which an attacker has occupied $n$ addresses using some subset $S \subset E, |S| = n$ of eviction set $E$, a following touch on address $\alpha$ may be in $S$ or not.

If $\alpha \in S$, there is no cache or per-line metadata to update, and the cache state does not change.

If $\alpha \notin S$, the random replacement algorithm will evict an address $\beta$.
This evicted address will occupy an unoccupied line with probability
$P(\beta \notin S) = \frac{a-n}{a}$.

Then, the probability occupying $n+1$ addresses in a cache set with $n$ addresses occupied is $P(n+1) = \frac{a-n}{a}$.
The corresponding expected value is $EV(n+1) = \frac{1}{P(n+1)} = \frac{a}{a-n}$

Then, the expected value of the total number of unique touches to occupy an eviction set is

% Can this be simplified? I can't think of anything right now
\begin{equation}
    \text{D}_{Random}(a) = \sum_{i=0}^{a-1}{\frac{a}{a-i}}
\end{equation}

\subsection{Preventing Eviction Set Occupation}
Attackers use eviction sets as attack primitives to guarantee that a target address is not present in the cache
so a victim's memory request of the target address must travel to DRAM and back.
Leaving at least one of the lines in a cache set open and unoccupied means that attackers have no guarantee that the target address is evicted.
Thus, to slow the use of an eviction set in occupying a cache set, we must only increase the difficulty of the eviction algorithm used.

\subsection{Latency and Energy Costs}

As the CPU implements eviction algorithms in hardware, the function of some software-based algorithms
may not be exactly reflected without inducing significant performance or complexity overhead.
For example, LRU can be implemented in a software cache to add, update, and remove objects in $O(1)$ time.
Although no such time complexity exists for algorithm circuitry that runs in parallel,
exact LRU is complex in CPU caches and involves lots of metadata per cache line and lots of circuitry to
determine an eviction candidate.
Excessive circuitry introduces extra latency by increasing the critical path of the circuit
and extra energy usage by requiring additional transistors.
The LRU algorithm is thus commonly simplified to TreePLRU in the CPU, which requires much less hardware complexity
but achieves a similar miss ratio.
Although the energy usage can be significant, modern CPU features like prefetching largely mitigate the latency factor of the algorithm.
Additionally, many modern CPUs punt the cache eviction calculation to after fulfilling the memory request.

Additionally, the cache - other than random replacement caches - may introduce registers for the cache metadata used in eviction.
However, the amount of metadata is much smaller than the amount of data in a cache set:
for an 8-way associative, 64-byte cache line, a TreePLRU replacement policy for the set uses
$\frac{7}{512 \cdot 8}=0.17\%$ as much space as the data itself.
Thus, within reason, we treat metadata usage as insignificant compared to the space and energy usage of the cache.

For the sake of simplicity and due to the lack of comparison against proprietary CPU cache implementations,
we analyze and evaluate our eviction algorithm qualitatively.
% I mean to say that we only check that the value is reasonable and move on,
% because that's all I can do

\subsection{Miss Rate Cost}

\section{Implementation}

As discussed in 3.3, due to the hardware requirements and parallel nature of circuits,
we implement our cache eviction algorithms as approximations of their software equivalents.
We implement and discuss multiple variants of each algorithm.

\subsection{SIEVE}

SIEVE, as implemented in software, does not translate well to hardware.
It relies on entirely sequential logic to determine an eviction candidate;
additionally, it conditionally returns the hand to the beginning of the cache array.
These two features imply the use of a register to keep track of the hand position,
constant updating to reflect its marching through the cache array
and its possibility of returning to the beginning of the array.
Along with an increased hardware footprint, this would result in immense energy usage,
especially compared to other algorithms like TreePLRU that select an eviction candidate
without sequential logic at all.

\subsubsection{Priority Encoder}

\subsubsection{Shift Register}

\subsection{TwoQ}

TwoQ does not rely on sequential logic nearly as much as SIEVE,
but there are multiple possible ways to implement the dual-queue structure.

\subsubsection{Single Tree}

\subsubsection{Dual Tree}

\subsubsection{Tag Bits}

\section{Experiment Methodology}

We evaluate our eviction algorithms on their minimum expected difficulty and their cache performance by the miss ratio on common workloads.
Additionally, we analyze the physical complexity of the algorithms for energy and space consumption.

\subsection{Eviction Difficulty Testing}

\subsection{Performance Testing}

We use the gem5 CPU simulator to run binaries and full Linux systems on a virtual CPU with custom architectures.

\section{Results}

\section{Discussion}


% PARSEC
% SPEC, if I can get it working
% Custom/synthetic workloads


%-------------------------------------------------------------------------------
\bibliographystyle{plain}
\bibliography{\jobname}

%%%%%%%%%%%%%%%%%%%%%%%%%%%%%%%%%%%%%%%%%%%%%%%%%%%%%%%%%%%%%%%%%%%%%%%%%%%%%%%%
\end{document}
%%%%%%%%%%%%%%%%%%%%%%%%%%%%%%%%%%%%%%%%%%%%%%%%%%%%%%%%%%%%%%%%%%%%%%%%%%%%%%%%

%%  LocalWords:  endnotes includegraphics fread ptr nobj noindent
%%  LocalWords:  pdflatex acks
